% $Header: /Users/joseph/Documents/LaTeX/beamer/solutions/generic-talks/generic-ornate-15min-45min.de.tex,v 90e850259b8b 2007/01/28 20:48:30 tantau $

\documentclass{beamer}

% Diese Datei enth�lt eine L�sungsvorlage f�r:


% - Vortr�ge �ber ein beliebiges Thema.
% - Vortragsl�nge zwischen 15 und 45 Minuten. 
% - Aussehen des Vortrags ist verschn�rkelt/dekorativ.



% Copyright 2004 by Till Tantau <tantau@users.sourceforge.net>.
%
% In principle, this file can be redistributed and/or modified under
% the terms of the GNU Public License, version 2.
%
% However, this file is supposed to be a template to be modified
% for your own needs. For this reason, if you use this file as a
% template and not specifically distribute it as part of a another
% package/program, I grant the extra permission to freely copy and
% modify this file as you see fit and even to delete this copyright
% notice. 



\mode<presentation>
{
  \usetheme{PaloAlto}
  \usecolortheme{seahorse}
  % oder ...
  
  % Items bleiben unsichtbar bis Sie aufgedeckt werden
  \setbeamercovered{invisible}
  % oder auch nicht
}


\usepackage[german]{babel}
% oder was auch immer

\usepackage[latin1]{inputenc}
% oder was auch immer

\usepackage{times}
\usepackage[T1]{fontenc}
% Oder was auch immer. Zu beachten ist, das Font und Encoding passen
% m�ssen. Falls T1 nicht funktioniert, kann man versuchen, die Zeile
% mit fontenc zu l�schen.



% % Mieser Hack f�r gr��ere Zeilenabst�nde
\newlength{\wideitemsep}
\setlength{\wideitemsep}{\itemsep}
\addtolength{\wideitemsep}{10pt}
\let\olditem\item
\renewcommand{\item}{\setlength{\itemsep}{\wideitemsep}\olditem}

\title[NiagarUI] % (optional, nur bei langen Titeln n�tig)
{NiagarUI}

\vskip10pt
\subtitle{Bachelor Projekt -- WS 15/16}
\vskip35pt
\author[] % (optional, nur bei vielen Autoren)
{Sebastian Schneider }
% - Der \inst{?} Befehl sollte nur verwendet werden, wenn die Autoren
%   unterschiedlichen Instituten angeh�ren.

%\institute % (optional, aber oft n�tig)
%{
%  \inst{1}%
%  Institut f�r Informatik\\
%  Universit�t Hier
%  \and
%  \inst{2}%
%  Institut f�r theoretische Philosophie\\
%  Universit�t Dort}
% - Der \inst{?} Befehl sollte nur verwendet werden, wenn die Autoren
%   unterschiedlichen Instituten angeh�ren.
% - Keep it simple, niemand interessiert sich f�r die genau Adresse.

\date{10. Mai 2016}


%\subject{Informatik}
% Dies wird lediglich in den PDF Informationskatalog einf�gt. Kann gut
% weggelassen werden.


% Falls eine Logodatei namens "university-logo-filename.xxx" vorhanden
% ist, wobei xxx ein von latex bzw. pdflatex lesbares Graphikformat
% ist, so kann man wie folgt ein Logo einf�gen:

\pgfdeclareimage[height=25pt, width=43pt]{logo}{UniLogo.jpg}
\logo{\pgfuseimage{logo}}



% Folgendes sollte gel�scht werden, wenn man nicht am Anfang jedes
% Unterabschnitts die Gliederung nochmal sehen m�chte.
%\AtBeginSubsection[]
%{
%  \begin{frame}<beamer>{Gliederung}
%    \tableofcontents[currentsection,currentsubsection]
%  \end{frame}
%}


% Falls Aufz�hlungen immer schrittweise gezeigt werden sollen, kann
% folgendes Kommando benutzt werden:

\beamerdefaultoverlayspecification{<+->}



\begin{document}

\begin{frame}
  \titlepage
\end{frame}

\section{Motivation f�r NiagarUI}

\subsection{NiagarUI}
\begin{frame}{Warum NiagarUI?}
	\begin{itemize}
		\item<+-> Vereinfachte Erstellung von Pl�nen
		
	\end{itemize}
\end{frame}
\begin{frame}{Warum NiagarUI?}
	Bisher wird niagarino der Plan per XML gef�ttert
\end{frame}
\begin{frame}{Warum NiagarUI?}
	\begin{itemize}
		\item Vereinfachte Erstellung von Pl�nen
		\item Veranschaulichung der Zusammenh�nge eines Planes
		
	\end{itemize}
\end{frame}
\begin{frame}{Warum NiagarUI?}
	Benutzerfreundliche spezifikation
\end{frame}

\begin{frame}{Warum NiagarUI?}
	\begin{itemize}
		\item Vereinfachte Erstellung von Pl�nen
		\item Veranschaulichung der Zusammenh�nge eines Planes
		\item NiagarUI ist ein cooler Name
		
	\end{itemize}
\end{frame}




\begin{frame}{Geplanter Rahmen des Projektes}
	\begin{itemize}
		\item 
		\item 
		\item 
	\end{itemize}
\end{frame}

\section{Umsetzung}
\subsection{Aufgetretene Schwierigkeiten}
\begin{frame}{Aufgetretene Schwierigkeiten}
	\begin{itemize}
		\item 
		\item 
		\item 
	\end{itemize}
\end{frame}

\begin{frame}{Einschr�nkung des Projektrahmens}
	\begin{itemize}
		\item 
		\item 
		\item 
	\end{itemize}
\end{frame}
\subsection{Proof of Concept}
\begin{frame}{Proof of Concept?}
	\begin{itemize}
		\item 
		\item 
		\item 
	\end{itemize}
\end{frame}

\begin{frame}{Proof of Concept!}
	\begin{itemize}
		\item Zwei Operatoren ausgew�hlt, um Machbarkeit, Prinzip und Weg zu veranschaulichen
		\item 
		\item 
		%TODO go into Code to show ParamDescriptions and Annotations 
	\end{itemize}
\end{frame}

\section{Ausblick}
\begin{frame}{Vom Proof of Concept zu funktionst�chtiger Umsetzung}
	
	\begin{itemize}
		\item Alle Operatoren integrieren
		\begin{itemize}[<+->]
			\item Annotationen zu allen Operatoren hinzuf�gen
			\item Zwei statische Methoden pro Operator erstellen die den Operator beschreiben
			\item Definieren wie spezielle Attribute (Functions, Group) implementiert und in XML �bersetzt werden sollen
		\end{itemize}
		
		\item Benutzeroberfl�che versch�nern
		
		\item Nutzerhandbuch
		
	\end{itemize}
\end{frame}

\begin{frame}{Ausblick}
	
	\begin{itemize}
		\item XML-Pl�ne einlesen
		
		\item Wenn wir Reflections bereits nutzen, warum dann nicht gleich Plan in niagarino ausf�hren?
		
		\item Dynamische Elemente mit einbauen -> Visualisierung der Arbeit der Operatoren direkt in GUI
		
	\end{itemize}
\end{frame}
\section{Example Workflow}
\begin{frame}{Beispiel}
	An der Wallstreet soll ...	

\end{frame}

\end{document}
